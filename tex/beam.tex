\documentclass[a4paper,11pt,twoside]{scrartcl}
\input{preamble}

\usepackage[utf8]{inputenc}
\usepackage[T1]{fontenc}
\usepackage{stmaryrd,url}
\RequirePackage[hyperref,table]{xcolor}



\renewcommand{\AdjointLetter}{\mathrm{T}}
\usepackage{lmodern}
\DeclareUnicodeCharacter{00B5}{\ensuremath{\mu}} % µ
\DeclareUnicodeCharacter{2264}{\ensuremath{\le}} % ≤
\DeclareUnicodeCharacter{2265}{\ensuremath{\ge}} % ≥
\DeclareUnicodeCharacter{22C5}{\!\cdot\!}
\usepackage{graphicx} % Required for inserting images


\title{Beam estimation}


\author{Ferréol Soulez }
\date{Juin 2023}

\begin{document}

\maketitle

\section{Direct estimation}

To prevent the use of fitting methods to estimate the beam size, the main idea is to consider that $(u,v)$ points were drawn from a 2D normal distribution. The  beam shape is extracted from the empirical covariance of this distribution.



We have $N$ points in the uv plane described by the vectors $\V u$ and $\V v$ (including the null frequency  $(u,v) = (0,0)$). We can define the modulation transfer function $h$ of the interferometer  as:
\begin{equation}
	\hat{h}(u, v) = \left\{
		\begin{array}{ll}
		1 &\text{if } u \in \V u \text{ and } v \in \V v\,,\\
		0 &\text{otherwise.}
	\end{array} \right.
\end{equation} 

The dirty beam is $h$  the Fourier transform of $\FT{h}$. One way to define the resolution of the interferometer is to compute the so called beam as half-width at half-maximum of a 2D Gaussian $g$ fitted on the central peak of the dirty beam. The covariance matrix of the Gaussian $\M C$ are given by
\begin{equation}
	\theta =  \argmin_{\M C} \Norm{ g(\M C) - h}^2 
\end{equation}
From  Parseval-Plancherel theorem, this can be also written in Fourier domain as
\begin{equation}
	\theta =  \argmin_{\M C} \Norm{\FT{g}(\M C) - \FT{h}}^2 
\end{equation}
where the $\FT{g}(\M C)$ is also a Gaussian of covariance matrix $\M D = \pi^{-2}\,\M C^{-1}$.

This covariance matrix $\M D$  can be approximated from the distribution of  the sampling point in the uv plane. As by construction $\Avg{\V u} = 0 $ and $\Avg{\V v} = 0$ we have:
\begin{equation}
	\M D = \frac{2}{2N+1}\left(
		\begin{array}{ll}
			 \sum_n u_n^2 	 &  \sum_k u_k \,v_k\\
			 \sum_k u_k \,v_k &  \sum_n v_n^2
		\end{array}
	\right)
\end{equation}
The $ \frac{2}{2N+1}$ factor is to  count twice all the (uv) points (due to the  symmetry of the uv plane) excepted the null frequency. 
In addition to its covariance matrix $\M C$, we can describe the beam with its principal angle $\theta$ and its half-widths at half maximum  $r_1$  and  $r_2$ along both the major and minor axes respectively.   These parameters can be extracted by the mean of the eigendecomposition:
\begin{equation}
	\M D = \M Q \, \M \Lambda \, \M{Q}\T
\end{equation}
where $\M Q$ and $\M \Lambda$ are the matrices of eigenvectors and eigenvalues respectively. The matrix $\M \Lambda$ is a diagonal matrix that contains the eigenvalues $(\lambda_1,\lambda_2)$ that are the variances along both axis. $\M Q$ is a rotation  by the principal angle. As  $\M C = \pi^{-2}\M D^{-1}$, its eigenvalues are $ \M \Lambda^{-1}$ and major and minor axis are inverted leading to a rotation of $\pi/2$ of the principal angle. As a result, these matrices can be expressed as a function of the beam parameters using as :
\begin{align}
	\M Q &= \left(
		\begin{array}{ll}
			\cos(\pi/2 - \theta) & -\sin(\pi/2 - \theta) \\
			\sin(\pi/2 - \theta) & \cos(\pi/2 - \theta)  
		\end{array}
		\right)\\
	\M \Lambda &=  \frac{\log(2)}{2\pi^2} \left(
		\begin{array}{ll}
		r_1^{-2}	&  0\\
		0 			&  r_2^{-2}
		\end{array}
		\right)\\
\end{align}
where $\frac{\sqrt{2\log(2)}}{2}$ is a factor to convert standard deviation to half-width at half-maximum.
The beam ellipse parameters are then:
\begin{align}
	r_1 = & \frac{\sqrt{2\log(2)}}{2\pi} \frac{1}{\sqrt{\lambda_1}}\\
	r_2 = & \frac{\sqrt{2\log(2)}}{2\pi}  \frac{1}{\sqrt{\lambda_2}}\\
	\theta = & \arctan\left(Q[1,1], Q[2,1]\right)
\end{align}
% then 
% \begin{equation}
% 	\V C =\frac{N}{\sum_n v_n^2 \sum_n v_n^2  -(\sum_k u_k \,v_k)^2} \left(
% 		\begin{array}{ll}
% 			\sum_n v_n^2 & -\sum_k u_k \,v_k\\
% 			-\sum_k u_k \,u_k & \sum_n v_n^2
% 		\end{array}
% 	\right)
% \end{equation}


In OImaging, we can plot the  beam by  applying  the transformation matrix $\M T = \frac{1}{\pi}\, \M{Q}\,\M \Lambda^{-1/2} $ to a circle of diameter $1$.


\section{Results}

See notebook \url{https://jovian.com/ferreols/beamexample} 
\end{document}
